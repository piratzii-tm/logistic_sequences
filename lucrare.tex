\documentclass[a4paper,12pt]{article}

\usepackage[romanian]{babel}

\title{\bf Secvențe logistice}


\author{\tt Poenaru Iulian, Raț Ioan Paul, Pop Iulian \thanks{Universitatea de Vest din Timișoara, Facultatea de Matematică și Informatică, specializarea Informatică, E-mail: iulian.poenaru03@e-uvt.ro, ioan.rat04@e-uvt.ro, iulian.pop03@e-uvt.ro}}

\date{}

\begin{document}
\maketitle

\begin{abstract}
O secvență care apare în ecologie ca un model pentru creșterea populației este definită de ecuația diferențială $p_n+1 = kp_n(1-p_n)$ unde $p_n$ măsoară mărimea populației $n$ a unei singure specii, astfel $0<=p_n<=1$. Acestea fiind zise, un ecologist ar fii interesat să prezică mărimea populației în raport cu trecerea timpului, astfel își pune anumite întrebări: Se va stabiliza mărimea populației la o anumită limită? Se va modifica într-o manieră circulară? Sau va da dovadă de un comportament total aleatoriu? La toate aceste întrebări vom încerca să răspundem în paginile care urmează, analizând seturi de date prezentate sub forma unor grafice.
\end{abstract}

\section{Introducere}



\end{document}